% Options for packages loaded elsewhere
\PassOptionsToPackage{unicode}{hyperref}
\PassOptionsToPackage{hyphens}{url}
%
\documentclass[
]{article}
\usepackage{lmodern}
\usepackage{amsmath}
\usepackage{ifxetex,ifluatex}
\ifnum 0\ifxetex 1\fi\ifluatex 1\fi=0 % if pdftex
  \usepackage[T1]{fontenc}
  \usepackage[utf8]{inputenc}
  \usepackage{textcomp} % provide euro and other symbols
  \usepackage{amssymb}
\else % if luatex or xetex
  \usepackage{unicode-math}
  \defaultfontfeatures{Scale=MatchLowercase}
  \defaultfontfeatures[\rmfamily]{Ligatures=TeX,Scale=1}
\fi
% Use upquote if available, for straight quotes in verbatim environments
\IfFileExists{upquote.sty}{\usepackage{upquote}}{}
\IfFileExists{microtype.sty}{% use microtype if available
  \usepackage[]{microtype}
  \UseMicrotypeSet[protrusion]{basicmath} % disable protrusion for tt fonts
}{}
\makeatletter
\@ifundefined{KOMAClassName}{% if non-KOMA class
  \IfFileExists{parskip.sty}{%
    \usepackage{parskip}
  }{% else
    \setlength{\parindent}{0pt}
    \setlength{\parskip}{6pt plus 2pt minus 1pt}}
}{% if KOMA class
  \KOMAoptions{parskip=half}}
\makeatother
\usepackage{xcolor}
\IfFileExists{xurl.sty}{\usepackage{xurl}}{} % add URL line breaks if available
\IfFileExists{bookmark.sty}{\usepackage{bookmark}}{\usepackage{hyperref}}
\hypersetup{
  pdftitle={Quantitative Ecology Project Summary},
  pdfauthor={Jacob Harrell},
  hidelinks,
  pdfcreator={LaTeX via pandoc}}
\urlstyle{same} % disable monospaced font for URLs
\usepackage[margin=1in]{geometry}
\usepackage{graphicx}
\makeatletter
\def\maxwidth{\ifdim\Gin@nat@width>\linewidth\linewidth\else\Gin@nat@width\fi}
\def\maxheight{\ifdim\Gin@nat@height>\textheight\textheight\else\Gin@nat@height\fi}
\makeatother
% Scale images if necessary, so that they will not overflow the page
% margins by default, and it is still possible to overwrite the defaults
% using explicit options in \includegraphics[width, height, ...]{}
\setkeys{Gin}{width=\maxwidth,height=\maxheight,keepaspectratio}
% Set default figure placement to htbp
\makeatletter
\def\fps@figure{htbp}
\makeatother
\setlength{\emergencystretch}{3em} % prevent overfull lines
\providecommand{\tightlist}{%
  \setlength{\itemsep}{0pt}\setlength{\parskip}{0pt}}
\setcounter{secnumdepth}{-\maxdimen} % remove section numbering
\ifluatex
  \usepackage{selnolig}  % disable illegal ligatures
\fi

\title{Quantitative Ecology Project Summary}
\author{Jacob Harrell}
\date{2/25/2021}

\begin{document}
\maketitle

Background:

Since its reintroduction back into the West Virginia landscape and the
very first Spring Gobbler season, the West Virginia Division of Natural
Resources has conducted a hunter survey each year for over 40
years.Anywhere from between 200 to 800 hunters participate in the survey
annually. This rather extensive data set has focused on hunter
efficiency and success over the years with regards to wild turkey, but
it has a huge potential to illustrate population dynamics throughout the
years for wild turkey. The surveys are conducted throughout the state of
West Virginia in every county, with some counties receiving more
pressure than others.

The data collected by the WVDNR includes data on:

\begin{enumerate}
\def\labelenumi{\arabic{enumi}.}
\tightlist
\item
  Location surveyed and whether on private or public land
\item
  Dates surveyed
\item
  Hours surveyed for each day surveyed
\item
  Total number of turkey observed
\item
  Proportion of turkeys observed that were gobblers
\item
  Proportion of turkeys observed that were hens
\item
  Number of turkeys shot at
\item
  Number of turkeys harvested
\item
  General weather conditions for dates surveyed.
\item
  Number of observed gobblers that were adults
\item
  Number of observed gobblers that were juveniles
\end{enumerate}

In the case of this data, hours spent hunting and location being hunted
would be the predictor variables that would determine whether turkeys
were observed. The number of observed turkeys would be a response
variable. Within the response variable.

The primary objectives for this project would be to:

\begin{enumerate}
\def\labelenumi{\arabic{enumi}.}
\tightlist
\item
  Create a model using the data to create a population index of turkey
  within West Virginia.
\item
  Using the model, identify trends in population over the years.
\item
  If possible, based on the data, discern sex ratios in populations over
  the years.
\item
  Using the model, be able to discern proportion of juvenile turkeys to
  adult turkeys within the population from year to year and identify any
  important trends.
\end{enumerate}

It may also be conceivable, depending on the data set, to determine
population dynamics on a county level. However, certain counties may
have less hunter participation than other counties and thus the data set
might not be robust enough in some areas to model population on the
county level.

\end{document}
